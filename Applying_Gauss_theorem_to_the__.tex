"Applying Gauss' theorem to the surface integral occurring in (\ref{dGM2}), we obtain:
\begin{equation}
\frac{\partial \rho_k}{\partial t} = - \div \rho_k v_k + \sum_{j=1}^r \nu_{kj} J_j,
\label{dGM3}
\end{equation}
since (\ref{dGM2}) is valid for any arbitrary volume V. This equation has the form of a so-called balance equation: the local change of the l.h.s. is equal to the negative divergence of the flow of $k$ and a source term giving the (production or destruction) of substance $k$.
Since the mass is conserved in each separate chemical reaction we have
\begin{equation}
\sum_{k=1}^n \nu_{kj}= 0 
\end{equation}
Summing equation ($\ref{dGM3}$) over all substances $k$, one obtains then the {\it law of conservation of mass}:
\begin{equation}
\frac{\partial \rho}{\partial t} = - \div \rho \vec{v}
\label{dGM5}
\end{equation}
where $\rho$ is the total density
\begin{equation}
\rho = \sum_{k=1}^n \rho_k
\end{equation}
and $\vec{v}$ the centre of mass ("barycentric") velocity
\begin{equation}
\vec{v} = \frac{1}{\rho} \sum_{k=1}^n \rho_k v_k.
\end{equation}
