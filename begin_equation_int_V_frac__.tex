\begin{equation}
\int^V \frac{\partial \rho_k}{\partial t}\, dV = - \int^\vec{\Omega} \rho_k\, v_k \cdot d\vec{\Omega} + \sum_{j=1}^r \int^V \nu_k J_i\, dV
\label{dGM2}
\end{equation}
where $d\vec{\Omega}$ is a vector with magnitude $d\Omega$ normal to the surface and counted positive from the inside to the outside. Furthermore $v_k$ is the velocity of $k$ and $\nu_{kj} J_j$  the production of $k$ per unit volume in the $j^{th}$ chemical reaction.  The quantity $\nu_{kj}$ divided by the molecular mass $M$ of the component $k$ is proportional to the stoichiometric coefficient with which $k$ appears in the chemical reaction $j$. The coefficients $\nu_{kj}$ are counted positive when the components $k$ appear in the second, negative when they appear in the first member of the reaction equation. The quantity $J_j$ is called the chemical reaction rate of reaction $j$. It represents a mass per unit volume and per unit time. the quantities $\rho_k$, $v_k$ and $J_j$ occurring in (\ref{dGM2}) are all functions of time and space coordinates.