%\textit{Oh, an empty article!}
\begin{quote}
\href{https://www.authorea.com/users/24891/articles/130803/_show_article}{Previous Chapter: Introduction  and Motivations}
\end{quote}

\section{Introduction}

DgM wrote: "Thermodynamics is based on two fundamental laws: the first law of thermodynamics or law of conservation of energy, and the second law of thermodynamics or entropy law".  I actually disagree with this. Equilibrium thermodynamics is. Non-equilibrium thermodynamics is not. Non equilibrium thermodynamics is based instead on the law of conservation of energy, the law of conservation of mass and the law of conservation of momenta. Entropy is just a part of the subdivision of internal energy of a body or of a system. The conservation of these quantities, in non relativistic conditions, is given for granted by the structure of the macroscopic space-time, and is particularly evident when mechanics is formulated in \href{https://en.wikipedia.org/wiki/Lagrangian_mechanics}{Lagrangian terms} [CITE]. This is the results of the \href{https://en.wikipedia.org/wiki/Noether%27s_theorem}{Nöther theorem}. 
Mechanics (but this is only a perspective on the matter: others can be obtained by using \href{https://en.wikipedia.org/wiki/Legendre_transformation}{Legendre transformations}) usually subdivides energy in three components, kinetic, potential; thermodynamics adds internal  energy and usually separates it in thermal, mechanical work, and chemical energy.
\begin{equation}
E = U  + K + V = (E_T + E_W + E_C) + K + V
\label{e_cons}
\end{equation}
where $E$ is the total energy, $K$ is Kinetic energy, $V$ is potential energy, $E_T$ is thermal energy, $E_W$ is (internal) work, $E_C$ is chemical energy. Is this total energy which is conserved, while no one of the other is alone, and up to a certain amount each one of the five terms that compare on the r.h.s. of equation (\ref{e_cons}) can dynamically exchange energy amounts with the others with certain limitations. 
History of thermodynamics is the history of which are these limitations. To be more precise, more than about energy in itself, any dynamics is interested in energy variation in time, and the absolute value of energy is define up to a (time) constant. 
Regarding the three terms in U, their time variation is decomposed in:
\begin{equation}
\frac{dE_T(\,)}{dt} = T \frac{dS(\,)}{dt}
\label{ET}
\end{equation}
where $T$ is the quantity usually know as \href{https://en.wikipedia.org/wiki/Temperature}{temperature} [CITE Dall'Amico, 2010], and $S(\,)$ is known as \href{https://en.wikipedia.org/wiki/Entropy}{entropy}. Parentheses indicate that entropy, in this view $S$ is a function of some other variable. Besides, 
\begin{equation}
\frac{dE_W(\,)}{dt} = - p \frac{dV(\,)}{dt}
\label{EW}
\end{equation}
where $p$ is the \href{https://en.wikipedia.org/wiki/Pressure}{pressure} that the system exerts on the rest of the world, and $V$ is its \href{https://en.wikipedia.org/wiki/Volume}{volume}. 
Finally
\begin{equation}
\frac{dE_C(\,)}{dt} = \sum_i^k \mu_i \frac{dN_i(\,)}{dt}
\label{EC}
\end{equation}
where the index $i$ covers all the chemical constituents (https://en.wikipedia.org/wiki/Phase_(matter)}{phases}) of the system, $\mu_i$ is called \href{https://en.wikipedia.org/wiki/Chemical_potential}{chemical potential}, and $N_i$ is the number of elementary constituents (molecules) of the systems.
Internal energy $U$ embeds internal form of kinetic and potential energy. 
"However, it is necessary to formulate these laws in a way suitable for the purpose at hands. In this chapter we shall be concerned with the first law of thermodynamics. Since we wish to develop a theory applicable to systems of which the properties are continuous functions of space coordinates and time, we shall give a local formulation  of the law of conservation of energy.As the local momentum and mass densities may change in time, we shall also need local formulations of the laws of conservation of momentum and mass.
In the following sections, these conservation laws will be written down for a multi component system in which chemical reactions may occur and on which conservative external forces are exerted. 
It may be remarked that the macroscopic conservation laws of matter, momentum and energy are, from a microscopic point of view, consequences of the mechanical laws governing the motions of the constituents particles of the system."
